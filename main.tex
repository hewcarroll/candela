\documentclass[12pt,a4paper]{article}

% Packages
\usepackage[utf8]{inputenc}
\usepackage[T1]{fontenc}
\usepackage{amsmath,amssymb,amsfonts}
\usepackage{graphicx}
\usepackage{booktabs}
\usepackage{hyperref}
\usepackage[margin=1in]{geometry}
\usepackage{setspace}
\usepackage{natbib}
\usepackage{float}
\usepackage{caption}
\usepackage{enumitem}
\usepackage{xcolor}
\usepackage{fancyhdr}

% Header/Footer
\pagestyle{fancy}
\fancyhf{}
\rhead{The Candle-Hour Standard}
\lhead{White Paper --- Draft}
\cfoot{\thepage}

% Hyperlink setup
\hypersetup{
    colorlinks=true,
    linkcolor=blue!70!black,
    citecolor=green!50!black,
    urlcolor=blue!70!black
}

% Spacing
\onehalfspacing

% Title
\title{
    \textbf{The Candle-Hour Standard} \\[0.5em]
    \large A Carbon-Energy Framework for Currency Valuation \\
    Derived from Thermodynamic First Principles
}
\author{
    Draft White Paper \\[0.5em]
    \textit{Originated as a thought experiment, February 2026}
}
\date{\today}

\begin{document}

\maketitle

\begin{abstract}
This paper proposes a utility-backed monetary unit---the \textbf{candle-hour}---defined as the thermal energy released by combusting European Pharmacopoeia-grade beeswax through a natural fiber wick over one hour. Beginning from the elemental role of carbon in organic chemistry and energy systems, we construct a unit of account rooted in thermodynamics rather than scarcity, institutional authority, or cryptographic proof. We then map the embodied carbon-energy cost of human survival in a developed economy to derive a minimum wage from first principles of physics and biology, constrained by occupational health research on sustainable work hours. The resulting figure---equivalent to approximately \$28 USD per hour for a single adult working a 30-hour week---converges with estimates produced by decades of conventional economic research using regression analysis, labor elasticity models, and regional cost-of-living indices. This convergence is not coincidental: it reflects the fact that the economy is, at its physical foundation, a system for distributing carbon-energy among human beings, and that the minimum wage debate is fundamentally a thermodynamics problem obscured by layers of financial abstraction. We present the full derivation, a conversion framework bridging candle-hours to US dollars and precious metals, a comparative analysis against gold-standard, fiat, and cryptocurrency systems, and an honest assessment of the framework's limitations.
\end{abstract}

\newpage
\tableofcontents
\newpage

%% ============================================================
%% SECTION 1: INTRODUCTION
%% ============================================================
\section{Introduction}
\label{sec:introduction}

What makes something valuable as money? The history of monetary systems reveals a progression of answers to this question, each reflecting the priorities of its era. Gold and silver served as monetary bases for millennia, deriving their status from a combination of rarity, durability, divisibility, and aesthetic appeal. The classical gold standard, formalized in the 19th century, anchored currency to a metal whose primary monetary virtue was that it did not corrode, could not be synthesized, and was difficult to extract---properties of \textit{scarcity}, not \textit{utility}.

Fiat currency, which replaced commodity backing across the 20th century, shifted the basis of monetary value to institutional authority. A dollar is worth a dollar because the United States government declares it legal tender and accepts it for tax obligations. This system offers extraordinary flexibility---governments can expand or contract the money supply in response to economic conditions---but at the cost of complete disconnection from physical reality.

Cryptocurrency, the most recent entrant, grounds monetary value in mathematical certainty: cryptographic proof-of-work or proof-of-stake, enforced by distributed consensus. Bitcoin's value derives from computational scarcity---the difficulty of solving hash functions---rather than physical scarcity or institutional backing.

None of these systems anchor value in what actually sustains human life. Gold cannot be eaten. Fiat paper has no caloric content. A Bitcoin hash solves no material problem. Each system is, in its own way, an abstraction layered over the physical reality that economies ultimately exist to manage: the distribution of energy and material resources among human beings.

This paper asks a simple question: \textit{What if currency were grounded in the thermodynamic reality of what it costs to keep a human being alive?}

We propose a monetary unit---the \textbf{candle-hour}---defined by the carbon-energy released during one hour of beeswax combustion through a natural fiber wick. This unit is:
\begin{itemize}[nosep]
    \item \textbf{Physically reproducible} anywhere with access to bees, flax, and fire;
    \item \textbf{Internationally standardizable} via existing pharmaceutical reference standards;
    \item \textbf{Thermodynamically grounded} in the combustion chemistry of carbon;
    \item \textbf{Historically resonant} with humanity's oldest timekeeping technology.
\end{itemize}

From this unit, we derive a weekly survival cost, a minimum wage, and a conversion framework to existing monetary systems---all from first principles, without recourse to econometric models, political negotiation, or historical wage data.

The central finding is striking: the minimum wage derived from pure thermodynamics converges with the figures produced by decades of conventional economic research. This convergence suggests that the minimum wage debate, long treated as a political and economic question, may be more accurately understood as a physics problem with a deterministic answer that has been obscured by the abstractions of modern finance.

%% ============================================================
%% SECTION 2: HISTORICAL CONTEXT
%% ============================================================
\section{Historical Context: A Century of Utility-Backed Proposals}
\label{sec:history}

The idea of anchoring currency to something with direct functional utility---rather than scarcity or institutional authority---is neither new nor fringe. It has attracted Nobel laureates, industrialists, and central bank architects for over a century, generating three distinct intellectual traditions.

\subsection{The Energy Tradition}

In 1921, Henry Ford proposed in the \textit{New York Tribune} that currency be replaced with ``units of power'' generated by a massive hydroelectric plant at Muscle Shoals, Tennessee. Ford argued that gold was controlled by international bankers who manipulated the money supply; energy, by contrast, was democratic and universally useful.

The idea found its fullest expression in the \textbf{Technocracy movement} of the 1930s, led by Howard Scott and the geophysicist M. King Hubbert. Their proposed system was radical: money would be replaced by \textbf{energy certificates} denominated in ergs or joules, distributed equally to the population, non-transferable, and expiring at the end of each accounting period. The theoretical justification was straightforward: since energy is the common input to all production, it constitutes the only scientific foundation for a monetary system.

\textbf{Frederick Soddy}, the Nobel Prize-winning chemist, provided the deepest theoretical framework in \textit{Wealth, Virtual Wealth and Debt} (1926). He distinguished between ``real wealth''---physical goods subject to thermodynamic entropy---and ``virtual wealth''---money and debt subject only to mathematical laws. His central insight remains devastating: financial debts grow exponentially at compound interest, but the real economy runs on exhaustible energy stocks. Four of his five policy proposals (abandoning the gold standard, floating exchange rates, countercyclical fiscal policy, and economic statistics bureaus) became standard practice within decades.

\textbf{Buckminster Fuller} extended the vision furthest, proposing ``World Kilowatt Dollars'' in \textit{Critical Path} (1981)---a global currency denominated in kilowatt-hours, distributed as universal basic income, enabled by an interconnected worldwide electrical grid.

\subsection{The Commodity-Basket Tradition}

Benjamin Graham---the father of value investing---considered his commodity reserve currency proposal his most important work. Across four publications (1933--1962), Graham proposed an international currency backed by a basket of at least 15 storable commodities, managed by a body that would buy when prices fell and sell when they rose, maintaining a $\pm$10\% price band.

Friedrich Hayek endorsed Graham's approach in a notable 1943 \textit{Economic Journal} article, arguing that a diversified commodity basket would eliminate gold's ``really serious objection''---supply inelasticity---while maintaining its virtues of automaticity. Keynes's \textbf{Bancor} proposal at Bretton Woods (1941--1944) took a different but related approach, proposing a supranational unit of account with symmetric adjustment mechanisms for both surplus and deficit nations.

\subsection{The Labor-Time Tradition}

Josiah Warren's Cincinnati Time Store (1827--1830) priced goods at cost plus a labor-time overhead, with a clock on the wall tracking transactions. Robert Owen's National Equitable Labour Exchange (1832--1834) attempted the same idea at scale and failed due to adverse selection---the exchange became a dumping ground for unsaleable goods.

\subsection{Historical Experiments}

Several utility-linked monetary experiments achieved notable results. The W\"{o}rgl experiment (Austria, 1932--1933) introduced demurrage currency and reduced unemployment by 16\% while it rose 19\% nationally. The \textbf{WIR Bank} (Switzerland, 1934--present) has operated a complementary business currency for over 90 years, serving approximately 50,000 Swiss SMEs with annual turnover of 1.43 billion CHF. Japan's \textbf{Fureai Kippu} (1995--present) operates a time-credit system for elder care across 374+ nonprofit organizations.

The pattern across these experiments reveals consistent success factors: addressing genuine needs, institutional formalization, and complementing rather than replacing national currency.

%% ============================================================
%% SECTION 3: UNIT DEFINITION
%% ============================================================
\section{The Candle-Hour: Unit Definition}
\label{sec:unit}

\subsection{Why Carbon}

Carbon occupies a unique position at the intersection of biology, energy, and industry. It is the backbone of every organic molecule, the stored energy in every fossil fuel, and the structural basis of materials from diamond to graphene. Plants fix atmospheric carbon through photosynthesis; combustion returns it. Nearly all traded commodities---food, fuel, plastics, textiles, lumber, pharmaceuticals---are carbon compounds.

A currency denominated in the energetic potential of carbon bonds unifies the energy and commodity traditions described in Section~\ref{sec:history} within a single elemental framework. It also creates an inherent bridge to climate accounting: carbon sequestration adds to the monetary base while carbon emission represents expenditure.

\subsection{Why Beeswax}

Beeswax is a biologically produced, carbon-dense substance with several properties that make it uniquely suitable as a monetary reference material:

\begin{enumerate}[nosep]
    \item \textbf{Historical connection to timekeeping.} Before mechanical clocks, combustion \textit{was} time. Candle clocks---marked candles burning at known rates---were among humanity's earliest timekeeping devices.
    \item \textbf{International standardization.} The European Pharmacopoeia (Ph. Eur.) defines mandatory quality standards for both yellow beeswax (\textit{Cera flava}) and white beeswax (\textit{Cera alba}), including drop point, acid value, ester value, and saponification number. Parallel standards exist in the USP, British Pharmacopoeia, and FAO/WHO Codex Alimentarius. ISO-certified reference standards are commercially available.
    \item \textbf{Chemical consistency.} Beeswax consists primarily of esters of fatty acids and long-chain alcohols, with an approximate formula of C$_{15}$H$_{31}$COOC$_{30}$H$_{61}$ and carbon content of approximately 80\% by mass. Its density is 0.95 g/cm$^3$ with a melting point of 62--64$^{\circ}$C.
    \item \textbf{Ecological resonance.} Beeswax is stored solar energy passed through photosynthesis, pollination ecology, and bee metabolism into a stable, energy-dense form. The monetary unit traces the full chain from sun to plant to insect to combustion.
\end{enumerate}

\subsection{Why a Natural Fiber Wick}

The wick material connects the unit to its historical and geographic roots. Flax (\textit{Linum usitatissimum}) was abundant in ancient Mesopotamia and was among the earliest cultivated fibers. The Sumerians developed the sexagesimal (base-60) system that produced our 60-minute hour. A flax-wicked beeswax candle thus embeds references to the origins of both timekeeping and numerical systems.

\subsection{Formal Definition}

\begin{quote}
\textbf{One candle-hour} is defined as the thermal energy released by the complete combustion of European Pharmacopoeia-grade beeswax (\textit{Cera flava} or \textit{Cera alba}) consumed during one hour of burning through a natural flax fiber wick of standard specification, at standard atmospheric pressure (101.325 kPa) and ambient temperature (25$^{\circ}$C), in still air.
\end{quote}

Under these conditions, a standard beeswax taper candle consumes approximately 5--7 grams of wax per hour, of which $\sim$80\% by mass is carbon. The heat of combustion of beeswax ranges from approximately 10 kcal/g (open-flame conditions) to 12.7 kcal/g (bomb calorimeter, complete combustion). Using a working estimate of 60 kcal per candle-hour:

\begin{table}[H]
\centering
\caption{Candle-hour thermodynamic parameters}
\begin{tabular}{ll}
\toprule
\textbf{Parameter} & \textbf{Value} \\
\midrule
Wax consumed per hour & 5--7 g \\
Carbon combusted per hour & 4--5.6 g \\
Energy released per candle-hour & $\sim$60 kcal (working estimate) \\
Beeswax carbon content & $\sim$80\% by mass \\
Beeswax density & 0.95 g/cm$^3$ \\
Beeswax melting point & 62--64$^{\circ}$C \\
\bottomrule
\end{tabular}
\label{tab:params}
\end{table}

\subsection{Currency Denomination}

The candle-hour serves as the base unit of the currency system in a decimal (base-10) structure:

\begin{align}
1 \text{ candle-hour} &= 1 \text{ cent (CH\textcent)} \\
100 \text{ candle-hours} &= \$1.00 \text{ CH} \\
1{,}000 \text{ candle-hours} &= \$10.00 \text{ CH}
\end{align}

%% ============================================================
%% SECTION 4: SURVIVAL BUDGET
%% ============================================================
\section{The Thermodynamic Survival Budget}
\label{sec:budget}

To derive a minimum wage, we first require the total embodied carbon-energy cost of sustaining one human life for one week in a developed economy. We use the United States as the reference case and adopt high-end estimates throughout to ensure the resulting floor is robust under worst-case conditions.

\subsection{Methodology}

Each category of survival need is traced back to its total carbon-energy inputs using embodied energy analysis. This includes not only the direct energy content of consumed goods but also the energy required for production, processing, transportation, and infrastructure maintenance.

\subsection{Category Estimates}

\subsubsection{Food: 2,300 candle-hours/week}

A human requires approximately 14,000 dietary kilocalories per week (2,000 kcal/day). However, the US food system requires approximately 7--10 calories of fossil energy input per calorie of food consumed, accounting for mechanized farming, synthetic fertilizer production (the Haber-Bosch process is extraordinarily energy-intensive), irrigation, harvest, processing, refrigeration, and transportation. Total carbon-energy throughput for food: $\sim$140,000 kcal/week $\div$ 60 kcal/CH $\approx$ 2,300 candle-hours.

\subsubsection{Shelter: 1,500 candle-hours/week}

Operating energy for a modest apartment (electricity, heating, cooling) runs approximately 50--70 kWh per week, equivalent to 43,000--60,000 kcal. Amortized embodied energy of construction materials---concrete (responsible for $\sim$8\% of global CO$_2$ emissions), steel, and lumber---adds approximately 20,000--30,000 kcal/week over the building's lifespan. Total: $\sim$90,000 kcal/week $\div$ 60 kcal/CH $\approx$ 1,500 candle-hours.

\subsubsection{Transportation: 2,000 candle-hours/week}

For a mixed transportation profile (some driving, some public transit) in a car-dependent US context, estimated at 60,000--120,000 kcal/week. Using the high-end estimate: $\sim$120,000 kcal/week $\div$ 60 kcal/CH $\approx$ 2,000 candle-hours.

\subsubsection{Water and Sanitation: 330 candle-hours/week}

Pumping, treatment, heating for bathing and cleaning, and wastewater processing: $\sim$20,000 kcal/week $\div$ 60 kcal/CH $\approx$ 330 candle-hours.

\subsubsection{Miscellaneous: 1,870 candle-hours/week}

Clothing (amortized over lifespan), healthcare (averaged across a healthy year), communications (phone, internet---infrastructure is energy-intensive), basic household supplies, and minimal social participation. Estimated as 30\% of the subtotal of the above categories: $0.30 \times 6{,}130 \approx 1{,}840$ candle-hours, rounded to 1,870.

\subsection{Total Survival Cost}

\begin{table}[H]
\centering
\caption{Weekly survival budget in candle-hours (single person, USA, high-end estimates)}
\begin{tabular}{lr}
\toprule
\textbf{Category} & \textbf{Candle-hours/week} \\
\midrule
Food (incl. full supply chain) & 2,300 \\
Shelter (operating + amortized construction) & 1,500 \\
Transportation (mixed, car-dependent) & 2,000 \\
Water and sanitation & 330 \\
Miscellaneous (clothing, health, comms, etc.) & 1,870 \\
\midrule
\textbf{Hard thermodynamic subtotal} & \textbf{$\sim$8,000} \\
\textbf{Margin for resilience and dignity} & \textbf{1,000} \\
\midrule
\textbf{Total weekly survival cost} & \textbf{9,000} \\
\bottomrule
\end{tabular}
\label{tab:budget}
\end{table}

The 1,000 candle-hour margin ($\sim$12.5\% above hard survival costs) accounts for unexpected needs, seasonal variation, and basic human dignity beyond bare biological survival. Annualized over 52 weeks:

\begin{equation}
\text{Annual survival cost} = 9{,}000 \times 52 = 468{,}000 \text{ candle-hours}
\end{equation}

%% ============================================================
%% SECTION 5: MINIMUM WAGE DERIVATION
%% ============================================================
\section{Minimum Wage Derivation}
\label{sec:wage}

\subsection{Inputs}

The minimum wage is derived from exactly two inputs:

\begin{enumerate}[nosep]
    \item \textbf{Weekly survival cost}: 9,000 candle-hours (Section~\ref{sec:budget})
    \item \textbf{Healthy work week}: 30 hours, based on occupational health research indicating that 30--32 hours per week is optimal for worker health, cognitive function, and sustained productivity
\end{enumerate}

\subsection{The Derivation}

\begin{equation}
\boxed{
w_{\min} = \frac{S}{H} = \frac{9{,}000 \text{ CH}}{30 \text{ hours}} = 300 \text{ candle-hours per hour}
}
\label{eq:minwage}
\end{equation}

where $w_{\min}$ is the minimum hourly wage, $S$ is the weekly survival cost in candle-hours, and $H$ is the maximum healthy weekly work hours.

In the candle-hour currency: $w_{\min} = \$3.00$ CH per hour.

\subsection{Properties of the Minimum}

This minimum wage has several notable properties:

\begin{enumerate}[nosep]
    \item \textbf{Derived, not legislated.} It follows from physical measurement and a health constraint, not political negotiation.
    \item \textbf{Universal base rate.} One person, one survival cost, one minimum. No adjustments for family size, which would reintroduce the political complexity the framework is designed to avoid.
    \item \textbf{Thermodynamic meaning.} Each hour of human labor must command at least 300 times the energy output of a beeswax candle burning for that same hour.
    \item \textbf{Divisibility.} 300 has factors of 2, 3, 4, 5, 6, 10, 12, 15, 20, 25, 30, 50, 60, 75, 100, and 150---facilitating mental arithmetic and clean pricing. Notably, $300 = 5 \times 60$, echoing the Sumerian sexagesimal system.
    \item \textbf{Market compatibility.} Everything above the floor is market-determined. The framework does not propose changing existing economic models, business structures, or market mechanisms.
\end{enumerate}

\subsection{Convergence with Conventional Economics}

Converting the candle-hour minimum wage to US dollars (see Section~\ref{sec:conversion}) yields approximately \textbf{\$28 USD per hour}. This figure falls within the range produced by:

\begin{itemize}[nosep]
    \item MIT Living Wage Calculator estimates for single adults (\$25--35/hour depending on location)
    \item Economic Policy Institute family budget calculations
    \item Various think tank analyses of ``living wage'' requirements
\end{itemize}

The convergence of a thermodynamic derivation with econometric estimates is the central finding of this paper and is analyzed in Section~\ref{sec:implications}.

%% ============================================================
%% SECTION 6: CONVERSION FRAMEWORK
%% ============================================================
\section{Conversion Framework}
\label{sec:conversion}

\subsection{Methodology}

The conversion rate between candle-hours and US dollars is derived by pricing the same weekly survival basket in both unit systems. This is not an arbitrary exchange rate---it emerges from mapping identical real goods and services in two different measurement frameworks.

\subsection{USD Survival Basket}

\begin{table}[H]
\centering
\caption{Weekly survival cost in USD (single person, USA, high-end estimates)}
\begin{tabular}{lr}
\toprule
\textbf{Category} & \textbf{USD/week} \\
\midrule
Food (groceries) & \$100 \\
Shelter (rent, \$1,400/month) & \$350 \\
Transportation (mixed) & \$225 \\
Water and sanitation & \$12 \\
Misc. (clothing, health, comms) & \$150 \\
\midrule
\textbf{Total} & \textbf{\$840} (rounded) \\
\bottomrule
\end{tabular}
\label{tab:usd_budget}
\end{table}

\subsection{The Conversion Rate}

\begin{equation}
\frac{9{,}000 \text{ CH}}{840 \text{ USD}} \implies 1 \text{ USD} \approx 10.7 \text{ candle-hours}
\end{equation}

\begin{equation}
\$1 \text{ CH} = 100 \text{ candle-hours} \approx \$9.33 \text{ USD}
\end{equation}

\subsection{Minimum Wage in USD}

\begin{equation}
w_{\min} = 300 \text{ CH/hour} = \$3.00 \text{ CH/hour} \approx \$28 \text{ USD/hour}
\end{equation}

For comparison, the current US federal minimum wage of \$7.25/hour equates to approximately 78 candle-hours per hour---\textbf{26\% of the thermodynamic survival floor}. A worker earning the federal minimum receives roughly one-quarter of the carbon-energy equivalence their continued existence requires.

\subsection{Precious Metal Valuations}

As of February 6, 2026:

\begin{table}[H]
\centering
\caption{Precious metal valuations in candle-hours}
\begin{tabular}{lrrrl}
\toprule
\textbf{Metal} & \textbf{USD/oz} & \textbf{CH/oz} & \textbf{\$ CH/oz} & \textbf{Human meaning} \\
\midrule
Gold & \$4,964 & 53,115 & \$531 & 5.9 weeks of survival \\
Silver & \$76 & 813 & \$8.13 & $\sim$15 hours of survival \\
\bottomrule
\end{tabular}
\label{tab:metals}
\end{table}

%% ============================================================
%% SECTION 7: COMPARATIVE ANALYSIS
%% ============================================================
\section{Comparative Analysis}
\label{sec:comparison}

\begin{table}[H]
\centering
\caption{Monetary systems compared across key dimensions}
\small
\begin{tabular}{lcccc}
\toprule
\textbf{Dimension} & \textbf{Gold Std.} & \textbf{Fiat} & \textbf{Crypto} & \textbf{Candle-Hour} \\
\midrule
Basis of value & Scarcity & Authority & Math & Utility \\
Intrinsic use & Minimal & None & None & Universal \\
Price stability & Mixed & Moderate & Poor & Theoretical \\
Scalability & Poor & Excellent & Limited & Moderate \\
Crisis resilience & Poor & Good & Poor & Mixed \\
Ecological feedback & None & None & Negative & Inherent \\
Manipulation resistance & Moderate & Poor & Moderate & High \\
Physical reproducibility & Limited & None & None & Universal \\
\bottomrule
\end{tabular}
\label{tab:comparison}
\end{table}

The candle-hour system's distinctive advantages are \textbf{intrinsic value} (the backing is something people actually need), \textbf{incentive alignment} (monetary expansion requires expansion of real energy production), and \textbf{ecological feedback} (carbon flows become visible in every transaction). Its distinctive disadvantages are the absence of empirical track record at scale and the theoretical challenges of pegging to a commodity whose production cost changes with technology.

%% ============================================================
%% SECTION 8: IMPLICATIONS
%% ============================================================
\section{Implications: Why the Numbers Converge}
\label{sec:implications}

The convergence of a thermodynamic derivation with conventional economic estimates is the central finding of this paper. We argue it is not coincidental but reflects a fundamental truth about economic systems.

\subsection{The Economy as Carbon-Energy Distribution}

All economic activity ultimately converts low-entropy carbon-energy (concentrated fuels, structured biomass) into goods, services, and high-entropy waste (dispersed CO$_2$, heat). Dollars are a unit invented to track this distribution. When the abstraction is stripped away and the distribution is measured directly in energy terms, the same answer emerges---because it was always the same question.

This perspective has deep roots in ecological economics, particularly the work of Nicholas Georgescu-Roegen (\textit{The Entropy Law and the Economic Process}, 1971), Herman Daly (steady-state economics), and Howard Odum (emergy analysis). What the candle-hour framework adds is not the insight itself but a specific, reproducible \textit{unit} that makes the insight operational.

\subsection{Conventional Economics Measures the Shadow}

Econometric approaches to the minimum wage---regression analysis, labor elasticity models, CPI baskets, regional cost-of-living indices---are sophisticated tools for estimating the same underlying physical reality: what does it cost in real resources to sustain a human life? They approach this question through prices, wages, and statistical inference. The candle-hour approach goes directly to the thermodynamic source.

The fact that both methods converge on the same figure ($\sim$\$25--35/hour range) suggests that decades of minimum wage research have been attempting to \textit{recover} an answer that the monetary system's abstraction layer obscures. The minimum wage debate is, at its physical foundation, a deterministic question with a calculable answer.

\subsection{Thermodynamic Deficit as Poverty}

In the candle-hour framework, poverty is not an abstract economic condition but a measurable \textbf{thermodynamic deficit}: a state in which a human being receives less carbon-energy equivalence than their biological and infrastructural survival requires. The current US federal minimum wage of \$7.25/hour delivers approximately 26\% of the thermodynamic survival floor. The remaining 74\% must be sourced from debt, government transfers, shared housing, skipped meals, or deferred healthcare---all of which are forms of energy borrowing that compound over time.

This reframing does not change the material reality of poverty, but it changes how we describe and measure it. You can argue about what constitutes a ``fair'' wage. You cannot argue with thermodynamics.

%% ============================================================
%% SECTION 9: LIMITATIONS
%% ============================================================
\section{Limitations and Open Questions}
\label{sec:limitations}

Intellectual honesty requires a clear accounting of what this framework does not solve.

\textbf{The measurement problem.} The marginal revolution of the 1870s established that economic value is subjective and ordinal. A candle-hour measures thermodynamic cost, not subjective utility. This is simultaneously a strength (providing an objective floor) and a limitation (not capturing all dimensions of economic value).

\textbf{Approximation in the survival budget.} Embodied energy estimates carry significant uncertainty. The food supply chain multiplier of 7--10$\times$ is a well-established range but varies by diet, geography, and agricultural method. The 30\% miscellaneous markup is a rough estimate. The framework's robustness depends on using high-end estimates throughout, which we have done.

\textbf{Geographic variation.} Survival costs differ dramatically by climate, infrastructure density, and local resource availability. The framework handles this naturally---different locations have different thermodynamic costs---but this paper calculates only one scenario (US, car-dependent suburb).

\textbf{Not a complete monetary system.} This paper defines a unit, derives a wage floor, and establishes a conversion framework. It does not address monetary policy, inflation management, banking, credit creation, or international trade. These are essential components of any functioning monetary system and represent future work.

\textbf{Snapshot conversion rate.} The USD conversion depends on current US prices, which change. The underlying candle-hour survival cost is more stable (thermodynamics change slowly), but the bridge to dollars would require periodic recalculation.

\textbf{Candle specification.} The exact burn rate depends on candle diameter, wick gauge, and ambient conditions. A formal metrological specification---analogous to the historical definition of the meter or candela---is needed to fix the unit precisely.

%% ============================================================
%% SECTION 10: CONCLUSION
%% ============================================================
\section{Conclusion}
\label{sec:conclusion}

We have defined a monetary unit from first principles: the thermal energy of carbon combustion in beeswax. We mapped human survival costs in that unit. We derived a minimum wage from thermodynamics and biology alone, constrained by a health-based work week. That wage converges with figures produced by decades of conventional economic research.

The framework did not require changing capitalism, inventing universal basic income, or assuming any alterations to existing economic models or business structures. It replaced the ruler, not the thing being measured. And the new ruler has hash marks that correspond to physical reality instead of political consensus.

Gold-backed money valued permanence and scarcity. Fiat money values institutional authority and flexibility. Cryptocurrency values mathematical certainty and individual sovereignty. The candle-hour proposes something different: \textit{we should value what sustains us.}

Whether this principle can be operationalized into a functioning monetary system at scale remains an open question. But the convergence demonstrated in this paper suggests that the physical foundations of economic value are more accessible than a century of monetary debate has assumed. The question the candle-hour poses---shouldn't our money reflect what actually matters?---will continue to demand answers.

%% ============================================================
%% REFERENCES
%% ============================================================
\newpage
\section*{References}
\addcontentsline{toc}{section}{References}

\begin{enumerate}[label={[\arabic*]}, nosep, leftmargin=*]
    \item Soddy, F. (1926). \textit{Wealth, Virtual Wealth and Debt}. George Allen \& Unwin.
    \item Georgescu-Roegen, N. (1971). \textit{The Entropy Law and the Economic Process}. Harvard University Press.
    \item Fuller, R.B. (1981). \textit{Critical Path}. St. Martin's Press.
    \item Graham, B. (1944). \textit{World Commodities and World Currency}. McGraw-Hill.
    \item Hayek, F.A. (1943). A Commodity Reserve Currency. \textit{Economic Journal}, 53(210/211), 176--184.
    \item Hayek, F.A. (1976). \textit{The Denationalisation of Money}. Institute of Economic Affairs.
    \item Keynes, J.M. (1943). Proposals for an International Clearing Union. British Government Publication, Cmd. 6437.
    \item Lietaer, B. (2001). \textit{The Future of Money}. Random House.
    \item Odum, H.T. (1996). \textit{Environmental Accounting: Emergy and Environmental Decision Making}. Wiley.
    \item Daly, H. (1991). \textit{Steady-State Economics}. Island Press.
    \item European Pharmacopoeia, 11th Edition. European Directorate for the Quality of Medicines (EDQM).
    \item Svecnjak, L. et al. (2019). Standard methods for \textit{Apis mellifera} beeswax research. \textit{Journal of Apicultural Research}, 58(2), 1--108.
    \item Stodder, J. (2009). Complementary credit networks and macro-economic stability: Switzerland's Wirtschaftsring. \textit{Journal of Economic Behavior \& Organization}, 72(1), 79--95.
    \item Fisher, I. (1920). \textit{Stabilizing the Dollar}. Macmillan.
    \item Pimentel, D. \& Pimentel, M. (2003). Sustainability of meat-based and plant-based diets and the environment. \textit{American Journal of Clinical Nutrition}, 78(3), 660S--663S.
\end{enumerate}

%% ============================================================
%% APPENDICES
%% ============================================================
\newpage
\appendix

\section{Detailed Thermodynamic Calculations}
\label{app:thermo}

\textit{[To be completed: Full derivation of carbon content, heat of combustion, and energy per candle-hour from beeswax chemical composition.]}

\section{Sensitivity Analysis}
\label{app:sensitivity}

\textit{[To be completed: How the minimum wage changes under different assumptions about burn rate, survival budget categories, and work week length.]}

\section{Household Economies of Scale}
\label{app:household}

As a real-world validation, we examined a US household of six persons with monthly expenses of approximately \$9,000 (rent \$3,100; utilities \$500; insurance \$500; transportation \$800; childcare \$1,600; groceries \$2,500). This converts to $\sim$24,075 candle-hours per week, yielding a per-person cost of $\sim$4,013 candle-hours---approximately 44.6\% of the single-person baseline. This reduction reflects genuine thermodynamic efficiencies in shared shelter, utilities, and transportation, and suggests that the single-person survival budget provides a conservative (high) estimate when applied to multi-person households.

\end{document}
